% Options for packages loaded elsewhere
\PassOptionsToPackage{unicode}{hyperref}
\PassOptionsToPackage{hyphens}{url}
%
\documentclass[
  12pt,
]{article}
\usepackage{amsmath,amssymb}
\usepackage{iftex}
\ifPDFTeX
  \usepackage[T1]{fontenc}
  \usepackage[utf8]{inputenc}
  \usepackage{textcomp} % provide euro and other symbols
\else % if luatex or xetex
  \usepackage{unicode-math} % this also loads fontspec
  \defaultfontfeatures{Scale=MatchLowercase}
  \defaultfontfeatures[\rmfamily]{Ligatures=TeX,Scale=1}
\fi
\usepackage{lmodern}
\ifPDFTeX\else
  % xetex/luatex font selection
\fi
% Use upquote if available, for straight quotes in verbatim environments
\IfFileExists{upquote.sty}{\usepackage{upquote}}{}
\IfFileExists{microtype.sty}{% use microtype if available
  \usepackage[]{microtype}
  \UseMicrotypeSet[protrusion]{basicmath} % disable protrusion for tt fonts
}{}
\makeatletter
\@ifundefined{KOMAClassName}{% if non-KOMA class
  \IfFileExists{parskip.sty}{%
    \usepackage{parskip}
  }{% else
    \setlength{\parindent}{0pt}
    \setlength{\parskip}{6pt plus 2pt minus 1pt}}
}{% if KOMA class
  \KOMAoptions{parskip=half}}
\makeatother
\usepackage{xcolor}
\usepackage[margin=1in]{geometry}
\usepackage{color}
\usepackage{fancyvrb}
\newcommand{\VerbBar}{|}
\newcommand{\VERB}{\Verb[commandchars=\\\{\}]}
\DefineVerbatimEnvironment{Highlighting}{Verbatim}{commandchars=\\\{\}}
% Add ',fontsize=\small' for more characters per line
\usepackage{framed}
\definecolor{shadecolor}{RGB}{248,248,248}
\newenvironment{Shaded}{\begin{snugshade}}{\end{snugshade}}
\newcommand{\AlertTok}[1]{\textcolor[rgb]{0.94,0.16,0.16}{#1}}
\newcommand{\AnnotationTok}[1]{\textcolor[rgb]{0.56,0.35,0.01}{\textbf{\textit{#1}}}}
\newcommand{\AttributeTok}[1]{\textcolor[rgb]{0.13,0.29,0.53}{#1}}
\newcommand{\BaseNTok}[1]{\textcolor[rgb]{0.00,0.00,0.81}{#1}}
\newcommand{\BuiltInTok}[1]{#1}
\newcommand{\CharTok}[1]{\textcolor[rgb]{0.31,0.60,0.02}{#1}}
\newcommand{\CommentTok}[1]{\textcolor[rgb]{0.56,0.35,0.01}{\textit{#1}}}
\newcommand{\CommentVarTok}[1]{\textcolor[rgb]{0.56,0.35,0.01}{\textbf{\textit{#1}}}}
\newcommand{\ConstantTok}[1]{\textcolor[rgb]{0.56,0.35,0.01}{#1}}
\newcommand{\ControlFlowTok}[1]{\textcolor[rgb]{0.13,0.29,0.53}{\textbf{#1}}}
\newcommand{\DataTypeTok}[1]{\textcolor[rgb]{0.13,0.29,0.53}{#1}}
\newcommand{\DecValTok}[1]{\textcolor[rgb]{0.00,0.00,0.81}{#1}}
\newcommand{\DocumentationTok}[1]{\textcolor[rgb]{0.56,0.35,0.01}{\textbf{\textit{#1}}}}
\newcommand{\ErrorTok}[1]{\textcolor[rgb]{0.64,0.00,0.00}{\textbf{#1}}}
\newcommand{\ExtensionTok}[1]{#1}
\newcommand{\FloatTok}[1]{\textcolor[rgb]{0.00,0.00,0.81}{#1}}
\newcommand{\FunctionTok}[1]{\textcolor[rgb]{0.13,0.29,0.53}{\textbf{#1}}}
\newcommand{\ImportTok}[1]{#1}
\newcommand{\InformationTok}[1]{\textcolor[rgb]{0.56,0.35,0.01}{\textbf{\textit{#1}}}}
\newcommand{\KeywordTok}[1]{\textcolor[rgb]{0.13,0.29,0.53}{\textbf{#1}}}
\newcommand{\NormalTok}[1]{#1}
\newcommand{\OperatorTok}[1]{\textcolor[rgb]{0.81,0.36,0.00}{\textbf{#1}}}
\newcommand{\OtherTok}[1]{\textcolor[rgb]{0.56,0.35,0.01}{#1}}
\newcommand{\PreprocessorTok}[1]{\textcolor[rgb]{0.56,0.35,0.01}{\textit{#1}}}
\newcommand{\RegionMarkerTok}[1]{#1}
\newcommand{\SpecialCharTok}[1]{\textcolor[rgb]{0.81,0.36,0.00}{\textbf{#1}}}
\newcommand{\SpecialStringTok}[1]{\textcolor[rgb]{0.31,0.60,0.02}{#1}}
\newcommand{\StringTok}[1]{\textcolor[rgb]{0.31,0.60,0.02}{#1}}
\newcommand{\VariableTok}[1]{\textcolor[rgb]{0.00,0.00,0.00}{#1}}
\newcommand{\VerbatimStringTok}[1]{\textcolor[rgb]{0.31,0.60,0.02}{#1}}
\newcommand{\WarningTok}[1]{\textcolor[rgb]{0.56,0.35,0.01}{\textbf{\textit{#1}}}}
\usepackage{graphicx}
\makeatletter
\newsavebox\pandoc@box
\newcommand*\pandocbounded[1]{% scales image to fit in text height/width
  \sbox\pandoc@box{#1}%
  \Gscale@div\@tempa{\textheight}{\dimexpr\ht\pandoc@box+\dp\pandoc@box\relax}%
  \Gscale@div\@tempb{\linewidth}{\wd\pandoc@box}%
  \ifdim\@tempb\p@<\@tempa\p@\let\@tempa\@tempb\fi% select the smaller of both
  \ifdim\@tempa\p@<\p@\scalebox{\@tempa}{\usebox\pandoc@box}%
  \else\usebox{\pandoc@box}%
  \fi%
}
% Set default figure placement to htbp
\def\fps@figure{htbp}
\makeatother
\setlength{\emergencystretch}{3em} % prevent overfull lines
\providecommand{\tightlist}{%
  \setlength{\itemsep}{0pt}\setlength{\parskip}{0pt}}
\setcounter{secnumdepth}{-\maxdimen} % remove section numbering
\usepackage{titlesec}
\titleformat{\section}{\Large\bfseries}{\thesection}{1em}{}
\titleformat{\subsection}{\large\bfseries}{\thesubsection}{1em}{}
\titleformat{\subsubsection}{\normalsize\bfseries}{\thesubsubsection}{1em}{}
\usepackage{float}
\usepackage{placeins}
\FloatBarrier
\usepackage{mdframed}
\usepackage{bookmark}
\IfFileExists{xurl.sty}{\usepackage{xurl}}{} % add URL line breaks if available
\urlstyle{same}
\hypersetup{
  pdftitle={AMProject\_Clean},
  pdfauthor={Laura Gullicksen, Erich Gozebina, Daria Palitzsch},
  hidelinks,
  pdfcreator={LaTeX via pandoc}}

\title{AMProject\_Clean}
\author{Laura Gullicksen, Erich Gozebina, Daria Palitzsch}
\date{23/05/2025}

\begin{document}
\maketitle

\subsection{1. Introduction}\label{introduction}

\#TODO: describe data choice -\textgreater{} Laura

\subsection{2. Data \& Descriptive
Analysis}\label{data-descriptive-analysis}

\#TODO: Explain the daily aggregation -\textgreater{}
7-day-trading-strategy -\textgreater{} daily makes more senses
-\textgreater{} Daria

\#TODO: Explain output and log choice -\textgreater{} Erich

Using log returns instead of simple (arithmetic) returns is a standard
practice in financial econometrics and modeling.

\begin{itemize}
\tightlist
\item
  Log returns are more symmetrically distributed and are better
  approximated by a normal distribution, especially for small time
  intervals (e.g., hourly/daily). This makes them more suitable for:

  \begin{itemize}
  \tightlist
  \item
    Linear models
  \item
    Hypothesis testing
  \item
    Machine learning regressors
  \end{itemize}
\item
  Because log returns are additive, they allow you to aggregate returns
  over multiple periods simply by summing simple return becomes
  undefined. Log return avoids this issue as long as prices are strictly
  positive, which is true for most financial assets (especially crypto).
\end{itemize}

\begin{Shaded}
\begin{Highlighting}[]
\CommentTok{\# Descriptive stats for prices and returns}
\NormalTok{summary\_stats }\OtherTok{\textless{}{-}}\NormalTok{ df\_daily }\SpecialCharTok{\%\textgreater{}\%}
  \FunctionTok{summarise}\NormalTok{(}
    \AttributeTok{n\_obs =} \FunctionTok{n}\NormalTok{(),}
    \AttributeTok{mean\_close =} \FunctionTok{mean}\NormalTok{(close\_price, }\AttributeTok{na.rm =} \ConstantTok{TRUE}\NormalTok{),}
    \AttributeTok{sd\_close =} \FunctionTok{sd}\NormalTok{(close\_price, }\AttributeTok{na.rm =} \ConstantTok{TRUE}\NormalTok{),}
    \AttributeTok{min\_close =} \FunctionTok{min}\NormalTok{(close\_price, }\AttributeTok{na.rm =} \ConstantTok{TRUE}\NormalTok{),}
    \AttributeTok{max\_close =} \FunctionTok{max}\NormalTok{(close\_price, }\AttributeTok{na.rm =} \ConstantTok{TRUE}\NormalTok{),}
    \AttributeTok{mean\_return =} \FunctionTok{mean}\NormalTok{(log\_return, }\AttributeTok{na.rm =} \ConstantTok{TRUE}\NormalTok{),}
    \AttributeTok{sd\_return =} \FunctionTok{sd}\NormalTok{(log\_return, }\AttributeTok{na.rm =} \ConstantTok{TRUE}\NormalTok{),}
    \AttributeTok{min\_return =} \FunctionTok{min}\NormalTok{(log\_return, }\AttributeTok{na.rm =} \ConstantTok{TRUE}\NormalTok{),}
    \AttributeTok{max\_return =} \FunctionTok{max}\NormalTok{(log\_return, }\AttributeTok{na.rm =} \ConstantTok{TRUE}\NormalTok{)}
\NormalTok{  )}

\NormalTok{summary\_stats\_long }\OtherTok{\textless{}{-}} \FunctionTok{as.data.frame}\NormalTok{(}\FunctionTok{t}\NormalTok{(summary\_stats))}
\FunctionTok{colnames}\NormalTok{(summary\_stats\_long) }\OtherTok{\textless{}{-}} \StringTok{"Value"}

\CommentTok{\# Add a column for metric names}
\NormalTok{summary\_stats\_long }\OtherTok{\textless{}{-}}\NormalTok{ tibble}\SpecialCharTok{::}\FunctionTok{rownames\_to\_column}\NormalTok{(summary\_stats\_long, }\AttributeTok{var =} \StringTok{"Statistic"}\NormalTok{)}

\CommentTok{\# Show result}
\NormalTok{summary\_stats\_long}
\end{Highlighting}
\end{Shaded}

\begin{verbatim}
##     Statistic         Value
## 1       n_obs  2.383000e+03
## 2  mean_close  9.148397e+00
## 3    sd_close  9.540702e+00
## 4   min_close  1.452550e-01
## 5   max_close  5.210000e+01
## 6 mean_return  1.594292e-03
## 7   sd_return  6.764975e-02
## 8  min_return -6.776430e-01
## 9  max_return  4.761717e-01
\end{verbatim}

\#TODO: create nice table output and explain -\textgreater{} Daria

\begin{Shaded}
\begin{Highlighting}[]
\CommentTok{\#TODO: Show them side{-}by{-}side }

\CommentTok{\# Plot closing price}
\CommentTok{\# ggplot(prices\_link, aes(x = date, y = close\_price)) +}
\CommentTok{\#   geom\_line(color = "steelblue") +}
\CommentTok{\#   labs(title = "Daily Close Price", x = "Date", y = "Price")}

\CommentTok{\# \# Plot returns}
\CommentTok{\# ggplot(prices\_link, aes(x = date, y = log\_return)) +}
\CommentTok{\#   geom\_line(color = "darkred") +}
\CommentTok{\#   labs(title = "Daily Log Returns", x = "Date", y = "Log Return")}
\end{Highlighting}
\end{Shaded}

\#TODO: create plots and explain output -\textgreater{} Erich

\begin{Shaded}
\begin{Highlighting}[]
\CommentTok{\# ACF plot of returns}
\FunctionTok{acf}\NormalTok{(}\FunctionTok{na.omit}\NormalTok{(df\_daily}\SpecialCharTok{$}\NormalTok{log\_return), }\AttributeTok{main =} \StringTok{"ACF of Daily Log Returns"}\NormalTok{)}
\end{Highlighting}
\end{Shaded}

\pandocbounded{\includegraphics[keepaspectratio]{AMProject_Clean_files/figure-latex/unnamed-chunk-4-1.pdf}}
\#TODO: make nicer plot: -\textgreater{} Daria

Interpretation: The autocorrelation function of daily log returns shows
no statistically significant linear dependence, indicating that past
returns do not linearly predict future returns. This supports the
weak-form Efficient Market Hypothesis. However, this does not rule out
the presence of exploitable patterns through non-linear or directional
indicators. Therefore, we adopt a momentum-based strategy, using the
sign of past multi-day returns to generate long or short trading
signals.

\subsection{3. Standard Model}\label{standard-model}

\#Momentum Signal Strategy

We define the 7-day momentum as the log return over the past 7 days: \[
\text{Momentum}_t = \log\left(\frac{P_t}{P_{t-7}}\right)
\]

The trading signal is then determined as: \[
\text{Signal}_t =
\begin{cases}
+1 & \text{if } \text{Momentum}_t > 0 \quad \text{(go long)} \\
-1 & \text{if } \text{Momentum}_t < 0 \quad \text{(go short)} \\
\;\;0 & \text{otherwise (no position)}
\end{cases}
\]

The strategy return is computed as: \[
r^{\text{strategy}}_{t+1} = \text{Signal}_t \cdot r_{t+1}
\] where \(r_{t+1} = \log\left(\frac{P_{t+1}}{P_t}\right)\) is the daily
log return.

\#TODO: insert standard model with momentum -\textgreater{} Erich

\#TODO: explain result of standard 7 day momentum strategy
-\textgreater{} Laura

\subsection{4. Extension}\label{extension}

\textbf{Extension of our OLS}

To enhance the predictive power of the benchmark model, we extend it by
incorporating a broader set of explanatory variables that capture not
only short- and medium-term price dynamics, but also market sentiment,
technical indicators, and inter-asset relationships. These include:

\begin{itemize}
  \item Momentum indicators over 3, 7, and 14 days,
  \item Lagged daily returns (1-day and 2-day),
  \item A 7-day rolling volatility measure,
  \item Technical indicators such as the 14-day Relative Strength Index (RSI), MACD value and histogram, Simple Moving Average difference, and Average True Range (ATR),
  \item Day-of-week dummy variables to capture potential calendar effects,
  \item BTC-based predictors: daily BTC return, 7-day BTC momentum, and 7-day BTC volatility.
\end{itemize}

The extended predictive regression model is specified as:

\[
r_{t+1} = \alpha + \sum_{h \in \{3,7,14\}} \beta_h \cdot \text{Momentum}_t^{(h)} + \gamma_1 \cdot r_t + \gamma_2 \cdot r_{t-1} + \delta \cdot \text{Volatility}_t^{(7)} + \sum_j \theta_j \cdot X_{t}^{(j)} + \varepsilon_{t+1}
\]

where \(X_t^{(j)}\) represents the set of technical indicators (RSI,
MACD, ATR, SMA), weekday dummies, and BTC-based predictors.

\begin{align*}
r_{t+1} &:= \log\left(\frac{P_{t+1}}{P_t}\right) \quad \text{(one-day-ahead LINK return)} \\
\text{Momentum}_t^{(h)} &:= \log\left(\frac{P_t}{P_{t-h}}\right) \quad \text{for } h \in \{3, 7, 14\} \\
\text{Volatility}_t^{(7)} &:= \text{std} \left( r_{t-6}, \ldots, r_t \right) \\
\text{BTC return}_t &:= \log\left(\frac{P^{\text{BTC}}_t}{P^{\text{BTC}}_{t-1}} \right) \\
\text{BTC Momentum}_t^{(7)} &:= \log\left(\frac{P^{\text{BTC}}_t}{P^{\text{BTC}}_{t-7}}\right) \\
\text{BTC Volatility}_t^{(7)} &:= \text{std} \left( r^{\text{BTC}}_{t-6}, \ldots, r^{\text{BTC}}_t \right)
\end{align*}

This model is estimated via Ordinary Least Squares (OLS) on the
in-sample period. By incorporating this rich feature set, we aim to
capture a range of return drivers including price trends, market
overreaction, volatility clustering, inter-market dependencies, and
behavioral biases tied to trading weekdays.

\#TODO: add ethereum data -\textgreater{} Erich

\#TODO: generate nicer latex table output of the regression results
-\textgreater{} Daria

\#TODO: Description and interpretation of output -\textgreater{} Laura

\textbf{Lasso Model}

To prevent overfitting and perform automatic variable selection, we
extend our linear modeling approach using the Lasso (Least Absolute
Shrinkage and Selection Operator). The Lasso adds a penalty term to the
standard OLS loss function, shrinking some coefficient estimates toward
zero. This results in a sparse model that may improve predictive
performance, particularly when dealing with multiple correlated
predictors.

The Lasso estimator is defined as the solution to the following
optimization problem:

\[
\hat{\beta}^{\text{lasso}} = \arg \min_{\beta_0, \beta} \left\{ \sum_{i=1}^{n} \left( y_i - \beta_0 - \sum_{j=1}^{p} x_{ij} \beta_j \right)^2 + \lambda \sum_{j=1}^{p} |\beta_j| \right\}
\]

where:

\begin{itemize}
  \item \( y_i \) is the target variable (e.g., one-day-ahead return),
  \item \( x_{ij} \) are the predictor variables,
  \item \( \beta_j \) are the coefficients,
  \item \( \lambda \geq 0 \) is the tuning parameter controlling the strength of the penalty.
\end{itemize}

As \(\lambda\) increases, more coefficients are shrunk toward zero. For
\(\lambda = 0\), the solution coincides with OLS.

We use 10-fold cross-validation to select the optimal \(\lambda\) that
minimizes the mean squared prediction error on held-out data.

\#TODO: generate nicer output

\#TODO: explain the results

\subsection{5. Forecasting \&
Backtesting}\label{forecasting-backtesting}

\textbf{in-sample testing}

\section*{4. In-Sample Testing}

To evaluate the performance of our predictive models, we begin by
conducting in-sample (IS) testing. This involves fitting each model on a
fixed training sample and evaluating how well the model explains
historical variation in the data.

We assess in-sample performance using the following criteria:

\begin{itemize}
  \item \textbf{Mean Squared Error (MSE)}: Measures the average squared difference between predicted and actual returns.
  \[
  \text{MSE} = \frac{1}{n} \sum_{i=1}^{n} (\hat{y}_i - y_i)^2
  \]
  
  \item \textbf{Adjusted \( R^2 \)}: Indicates the proportion of variance explained by the model, adjusted for the number of predictors.
  \[
  R_{\text{adj}}^2 = 1 - \frac{\text{RSS}/(n - p - 1)}{\text{TSS}/(n - 1)}
  \]

  \item \textbf{Directional Accuracy}: The fraction of times the predicted direction matches the actual direction of returns.
  \[
  \text{Accuracy} = \frac{1}{n} \sum_{i=1}^{n} \mathbb{1} \left( \text{sign}(\hat{y}_i) = \text{sign}(y_i) \right)
  \]
\end{itemize}

These metrics are computed for all three models:

\begin{enumerate}
  \item Benchmark (7-day momentum only),
  \item Extended linear model with multiple features,
  \item Lasso-regularized regression with automatic feature selection.
\end{enumerate}

\#TODO: interpret results

\textbf{Out-of-sample testing:}

\#TODO: review code, does not work at the moment

\#TODO: Evaluate: o Sharpe ratio o Cumulative return o OOSR2 o Hit rate
(how often you correctly predict direction)

\#TODO: include Transaction fees as extra path

\subsection{6. Conclusion}\label{conclusion}

\end{document}
